\documentclass[12pt,letterpaper]{article}
\usepackage[utf8]{inputenc}
\usepackage[spanish]{babel}
\usepackage{graphicx}
\usepackage[left=2cm,right=2cm,top=2cm,bottom=2cm]{geometry}
\usepackage{graphicx} % figuras
% \usepackage{subfigure} % subfiguras
\usepackage{float} % para usar [H]
\usepackage{amsmath}
%\usepackage{txfonts}
\usepackage{stackrel} 
\usepackage{multirow}
\usepackage{enumerate} % enumerados
\renewcommand{\labelitemi}{$-$}
\renewcommand{\labelitemii}{$\cdot$}
% \author{}
% \title{Caratula}
\begin{document}

% Fancy Header and Footer
% \usepackage{fancyhdr}
% \pagestyle{fancy}
% \cfoot{}
% \rfoot{\thepage}
%

% \usepackage[hidelinks]{hyperref} % CREA HYPERVINCULOS EN INDICE

% \author{}
\title{Caratula}

\begin{titlepage}
\begin{center}
\large{UNIVERSIDAD PRIVADA-DE-TACNA}\\
\vspace*{-0.025in}
\begin{figure}[htb]
\begin{center}
\includegraphics[width=8cm]{./Imagenes/logo}
\end{center}
\end{figure}
\vspace*{0.15in}
INGENIERIA DE SISTEMAS  \\

\vspace*{0.5in}
\begin{large}
TITULO:\\
\end{large}

\vspace*{0.1in}
\begin{Large}
\textbf{TRABAJO FINAL DE UNIDAD I} \\
\end{Large}

\vspace*{0.3in}
\begin{Large}
\textbf{CURSO:} \\
\end{Large}

\vspace*{0.1in}
\begin{large}
BASE DE DATOS II\\
\end{large}

\vspace*{0.3in}
\begin{Large}
\textbf{DOCENTE(ING):} \\
\end{Large}

\vspace*{0.1in}
\begin{large}
 Patrick Cuadros Quiroga\\
\end{large}

\vspace*{0.2in}
\vspace*{0.1in}
\begin{large}
Integrantes: \\
\begin{flushleft}
Arlyn Cotrado Coaquira		\hfill	(2016054466) \\
Yaneth Virginia Aquino Huallpa		\hfill	(2017059286) \\
Sharon Sosa Bedoya            	\hfill	(2016054460) \\
Marlo Villegas Arando	\hfill	(2015053890) \\
\end{flushleft}
\end{large}
\end{center}

\end{titlepage}


\tableofcontents % INDICE
\thispagestyle{empty} % INDICE SIN NUMERO
\newpage
\setcounter{page}{1} % REINICIAR CONTADOR DE PAGINAS DESPUES DEL INDICE

%%------------------------------------------------------------------------------------
\begin{abstract}

Balanced Scorecard y Business model canvas son técnicas de análisis empresarial . Ambas técnicas son útiles para mejorar el desempeño organizacional. Pero sus aplicaciones difieren. Ambos se pueden usar junto con los indicadores clave de rendimiento para monitorear y mejorar el rendimiento de la organización. Vamos a entender tanto la técnica en detalle.


\begin{center}
\textbf{Abstract}
\end{center}
Balanced Scorecard and Business model canvas are business analysis techniques. Both techniques are useful to improve organizational performance. But their applications differ. Both can be used together with the key performance indicators to monitor and improve the performance of the organization. We will understand both the technique and the detail.

\end{abstract}

\section{INTRODUCCIÓN} 
\section{TITULO}
\section{AUTORES}
\section{PLANTEAMIENTO DEL PROBLEMA}
	\subsection{Problema}
	\subsection{Justificación}
	\subsection{Alcance}
\section{OBJETIVOS}
	\subsection{General}
	\subsection{Específicos}
\section{REFERENTES TEÓRICOS}
\section{DESARROLLO DE LA PROPUESTA}
	\subsection{Tecnología de información}
	\subsection{Metodología, técnicas usadas}
\section{CRONOGRAMA}



%---------------------------------------------------------------------------------------
\cite{1}
\bibliographystyle{plain}
\bibliography{Bibliografia}

\end{document}
